\documentclass[Handouts]{beamer} % "Beamer" is a word used in Germany to mean video projector. 

%\usetheme{Warsaw}
%\usetheme{Antibes}
%\usetheme{Bergen}
%\usetheme{Berkeley}
%\usetheme{Berlin}
%\usetheme{Copenhagen}
%\usetheme{Darmstadt}
%\usetheme{Dresden}
%\usetheme{Frankfurt}
%\usetheme{Goettingen}
%\usetheme{Hannover}
%\usetheme{Ilmenau}
%\usetheme{JuanLesPins}
%\usetheme{Luebeck}
%\usetheme{Madrid}
%\usetheme{Malmoe}
%\usetheme{Marburg}
%\usetheme{Montpellier}
%\usetheme{PaloAlto}
%\usetheme{Pittsburgh}
%\usetheme{Rochester}
%\usetheme{Singapore}
%\usetheme{Szeged}
%\usetheme{boxes}
%\usetheme{default}
\usetheme{CambridgeUS}
% Search online for beamer themes to find your favorite or use the Berkeley theme as in this file.

%\usecolortheme{beaver}
%\usecolortheme{default}
%\usecolortheme{albatross}
%\usecolortheme{crane}
%\usecolortheme{dolphin}
%\usecolortheme{dove}
%\usecolortheme{fly}
%\usecolortheme{lily}
%\usecolortheme{orchid}
%\usecolortheme{rose}
%\usecolortheme{seagull}
%\usecolortheme{seahorse}
%\usecolortheme{whale}
%\usecolortheme{wolverine}



\usepackage{color} % It may be necessary to set PCTeX or whatever program you are using to output a .pdf instead of a .dvi file in order to see color on your screen.
\usepackage{graphicx} % This package is needed if you wish to include external image files.
\usepackage{url}
\usepackage{amsmath}
\usepackage{mathrsfs}
%\usepackage{minted}
%\usepackage{media9}
%\usepackage{listings}

%\lstset{ %
%	backgroundcolor=\color{white},   % choose the background color; you must add \usepackage{color} or \usepackage{xcolor}
%	basicstyle=\footnotesize,        % the size of the fonts that are used for the code
%	breakatwhitespace=false,         % sets if automatic breaks should only happen at whitespace
%	breaklines=true,                 % sets automatic line breaking
%	captionpos=b,                    % sets the caption-position to bottom
%	commentstyle=\color{mygreen},    % comment style
%	deletekeywords={...},            % if you want to delete keywords from the given language
%	escapeinside={\%*}{*)},          % if you want to add LaTeX within your code
%	extendedchars=true,              % lets you use non-ASCII characters; for 8-bits encodings only, does not work with UTF-8
%	frame=single,                    % adds a frame around the code
%	keepspaces=true,                 % keeps spaces in text, useful for keeping indentation of code (possibly needs columns=flexible)
%	keywordstyle=\color{blue},       % keyword style
%	language=python,                 % the language of the code
%	morekeywords={*,...},            % if you want to add more keywords to the set
%numbers=left,                    % where to put the line-numbers; possible values are (none, left, right)
%	numbersep=5pt,                   % how far the line-numbers are from the code
%	numberstyle=\tiny\color{mygray}, % the style that is used for the line-numbers
%	rulecolor=\color{black},         % if not set, the frame-color may be changed on line-breaks within not-black text (e.g. comments (green here))
%	showspaces=false,                % show spaces everywhere adding particular underscores; it overrides 'showstringspaces'
%	showstringspaces=false,          % underline spaces within strings only
%	showtabs=false,                  % show tabs within strings adding particular underscores
%	stepnumber=1,                    % the step between two line-numbers. If it's 1, each line will be numbered
%	stringstyle=\color{mymauve},     % string literal style
%	tabsize=2,                       % sets default tabsize to 2 spaces
%	title=\lstname                   % show the filename of files included with \lstinputlisting; also try caption instead of title
%}




\title{IoT with python and RPi}
\author{Dr. Sandeep Nagar} 
%\institute{G. D. Goenka University}
\date{\today} 

\begin{document}
\begin{frame}{IoT}

\centering
\Large{IoT with python and Raspberry Pi}\\
\textbf{PyDelhi 2016}
	
\end{frame}

\begin{frame}{Instructor}
	\begin{figure}
		\centering
		\includegraphics[width=4cm,height=4cm,keepaspectratio]{me}
	\end{figure}
	
	\centering
	\textcolor{blue}{Dr. Sandeep Nagar} \\
	M.Sc. Physics (MSU, Vadodara) \& PhD in Material Science \\ (Department of Material Science and Engineering, KTH, Sweden) \\
	contact e-mail: \textcolor{red}{sandeep.nagar@gmail.com}
\end{frame}

\begin{frame}{Why would I do IoT?}
	\begin{itemize}
		\item Its for everybody!
		\item Started just for fun
		\item Some serious experimentation
		\item Making scientific instruments
		\item Internet control gives multi-functionality to experiments
	\end{itemize}
\end{frame}

\begin{frame}{Outline of workshop}
	\begin{enumerate}
		\item Introduction to Raspberry Pi ($30$ Minutes)
		\begin{itemize}
			\item Various parts
			\item Installing OS
		\end{itemize}
		\item Accessing GPIO pins ($30$ minutes)
		\begin{itemize}
			\item Writing to GPIO pins
			\item Reading from GPIO pins
		\end{itemize}
		\item IoT with RPi ($30$ minutes)
		\begin{itemize}
			\item Running RPi headless 
			\item Adding sensors
			\item Interacting with data generated using IoT device
		\end{itemize}
	\end{enumerate}
\end{frame}

\begin{frame}{title}
	content...
\end{frame}
\begin{frame}{Python on RPi}
	\begin{itemize}
		\item Used to program to access pins and process
		\item Can use any language!
		\item $C$ and $C++$ requires a compiler called \textbf{gcc} which is pre-installed
		\item Python interpreter is also pre-installed
		\item We shall use Python $3$ instead of Python $2$ here because most packages for RPi usage are written in Python $3$
	\end{itemize}
\end{frame}

\begin{frame}{Writing python code}
	\begin{itemize}
		\item Python programming Environments
		\begin{itemize}
			\item IDE
			\begin{itemize}
				\item Combines the facilities of interpreter and text editor
				\item Default IDE is IDLE
				\item Invoke: Menu $\rightarrow$ Programming $\rightarrow$ Python
				\item Select Python $3$
			\end{itemize}
			\item Text-editor and interpreter separately
			\begin{itemize}
				\item Use \textbf{nano} to write code (ex: hello.py)
				\item Execute the program by typing \textbf{python3 hello.py}
			\end{itemize}
		\end{itemize}
	\end{itemize}
\end{frame}

\begin{frame}{GPIO configuration}
	\begin{figure}
		\centering
		\includegraphics[width=4cm,height=7cm,keepaspectratio]{gpio}
	\end{figure}
\end{frame}

\begin{frame}{GPIO}
	\begin{itemize}
		\item Dedicated power and ground pins
		\begin{itemize}
			\item $3.3V (1,17)$
			\item $5V (2.4)$
			\item GND $(6,9,14,20,30,39)$
		\end{itemize}
		\item GPIO = General Purpose Input Output
		\item Make pins \textit{input} or \textit{output} pins as per choice
		\item There are two numbering systems
		\begin{itemize}
			\item pin number based on location
			\item Pin number given as GPIO1, GPIO2 etc.
		\end{itemize}
	\end{itemize}
\end{frame}

\begin{frame}{Protocol pins}
	\begin{itemize}
		\item \textbf{I2C}
		\begin{itemize}
			\item Pin no.3 (GPIO2) = SDA1 I2C
			\item Pin No.5 (GPIO3) = SCL1 I2C
			\item Serial communication protocol between two chips relatively closely placed and need to share a clock
			Two wire protocol (SDA = sends Data signal, SCL = sends Clock signal)
		\end{itemize}
		\item If there are several I2C compatible devices, one can connect their SDA and SCL lines together for serial communication between them
	\end{itemize}
\end{frame}

\begin{frame}{Protocol Pins}
	\begin{itemize}
		\item \textbf{SPI}
		\begin{itemize}
			\item 19 (GPIO10) = MOSI (Master Out Slave In)
			\item 21 (GPIO9) = MISO (Master In Slave Out)
			\item 23 (GPIO11) = SCLK (S Clock)
			\item 24 (GPIO8) = CE01 (Chip Enable)
			\item 26 (GPIO7) = CE02 (Chip Enable)
		\end{itemize}
	\end{itemize}
\end{frame}


\end{document}